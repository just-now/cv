% Created 2023-06-08 Thu 23:29
% Intended LaTeX compiler: pdflatex
\documentclass[8pt,a4paper]{article}
\usepackage[utf8]{inputenc}
\usepackage[T1]{fontenc}
\usepackage{graphicx}
\usepackage{longtable}
\usepackage{wrapfig}
\usepackage{rotating}
\usepackage[normalem]{ulem}
\usepackage{amsmath}
\usepackage{amssymb}
\usepackage{capt-of}
\usepackage{hyperref}
\pagenumbering{gobble}
\usepackage{array, xcolor, bibentry}
\usepackage[margin=2cm]{geometry}
\usepackage{titlesec}
\usepackage{titling}
\definecolor{lightgray}{gray}{0.8}
\newcolumntype{L}{>{\raggedleft}p{0.14\textwidth}}
\newcolumntype{R}{p{0.78\textwidth}}
\newcommand\VRule{\color{lightgray}\vrule width 0.5pt}
\renewcommand{\hline}{}
\setlength{\parindent}{0pt}
\titleformat{\section}{\bfseries}{}{0em}{}[\titlerule]
\titlespacing{\section}{0pt}{5pt}{5pt}
\renewcommand{\maketitle}{\begin{center}{\LARGE\bfseries \theauthor} \vspace{5pt} \smallbreak \thetitle \end{center}}
\author{Anatoliy Bilenko}
\date{\today}
\title{}
\hypersetup{
 pdftitle={Anatoliy Bilenko},
 pdfkeywords={},
 pdfsubject={},
 pdfcreator={Emacs 28.2 (Org mode 9.5.5)},
 pdflang={English}}
\begin{document}

\maketitle



\section*{CONTACTS}
\label{sec:org2d18999}
\begin{itemize}
\item anatoliy.bilenko@gmail.com
\item \href{https://www.linkedin.com/in/anatoliy-bilenko-4367055/}{linkedin}
\item Location: Europe
\end{itemize}

\section*{BACKGROUND}
\label{sec:org9d23cda}
\begin{itemize}
\item Linux systems programmer with some experience in Linux kernel;
\item Leading and mentoring people;
\item Fluent in C/C++, assembly, bash, python, Go;
\item Distributed systems;
\item Telecommunications and protocols;
\item Discrete event simulation;
\item Solution-wide performance debugging;
\item Statistics, Data mining, Machine Learning, Computer vision, DSP;
\item Open source contributor.
\end{itemize}


\section*{FreedomFI/Nova-Labs -- LEAD SWE, ARCHITECT}
\label{sec:org9087acf}
\section*{Feb 2022 - May 2023}
\label{sec:org378e4ca}

\url{https://github.com/magma/}

Lead development of Wi-Fi offload solution.
Involved into develpement of telecom core frameworks such as magma
designed for enabling experienced users to run own mini-telecom
operator.

Software product design, including requirements gathering, use-case
definition, milestone planning, and integration. Communication of
technical decisions, development plans to software developers,
product management.

\begin{itemize}
\item Wi-Fi offload team lead;
\item Designed and implemented smoothless LTE to Wi-Fi offloading;
\item Designed and prototyped security workflow for Wi-Fi hotspots,
including implementation of TrustZone kernel module and trustlet;
\item Prototyped inbound roaming feature.
\end{itemize}

Technologies: EAP-AKA, EAP-TLS, Passpoint 2.0, Radius, Radsec,
Diameter, M-TLS, Wi-Fi, sWX, S8, S5, S11, Gx, Gy, OpenWiFi TIP,
Radiator, Radsecoroxy, OpenWRT, TrustZone, openssl, Open vSwitch,
OpenFlow, AWS, docker, k8s, HashiCorp vault (CA), YateHSS, C, C++,
Go, python.


\section*{SEAGATE/XYRATEX -- LEAD SWE, ARCHITECT}
\label{sec:org223daaa}
\section*{2011-2022}
\label{sec:org5c9ee24}

Distributed object store. Remote work, globally distributed team.

\url{https://github.com/Seagate/cortx-motr}

Involved into develpement of distributed object storage system
designed for great efficiency, massive capacity, and high
HDD-utilization.

Technologies: C, asm, python, bash, various cluster hardware, network
	      and storage controllers such as Mellanox Connectx-5,
	      Seagate Exos X 5U84, Supermicro TR4, PCI busses, HBAs,
	      simd.

\begin{itemize}
\item 2020 - 2022: Distributed transaction management team lead (4 engineers);
\item 2019 - 2020: Performance team lead (8-15 engineers);
\item 2018 - 2022: A part of architectural group. Software product
design, including requirements gathering, use-case definition,
milestone planning, and integration with complementary
subsystems. Communication of technical decisions, development plans
to software developers, team leads, product management.
\item RAID NxPxK library design and implementation;
\item Distributed configuration component implementation;
\item Preemptive locking primitive design and implementation;
\item "Parity-math" component as a part of SNS-repair (from HLD to TESTING);
\item HLD and DLD development of "rpc-layer" component;
\item Misc. tasks from CODE to TESTING like lib improvement, etc;
\item Transaction engine component design and implementation;
\item Designed implementation plan for the transaction engine integration
(overall work was around \textasciitilde{}5k hours);
\item Designed and implemented btree persistent structure;
\item Designed approaches (based on existing but with significant project
specifics) allowing to increase parallelism level for storage
structures of transaction engine;
\item Performance tuning and optimization in different contexts: from
application to system-wide;
\item Page Daemon design and implementation;
\item Designed implementation plan for the Page Daemon component
integration (an alternative was to rewrite the whole project code
base);
\item Distributed transaction manager detailed level design (high level
design was proposed by the architect);
\item Different kinds of library algorithms and improvements mostly
related to parallel and asynchronous programming (ex:
parallel\_for() implementation, async termination implementation,
etc);
\item Different kinds of scientific work like system performance modeling
(queuing theory);
\item Designed and implemented distributed profiler;
\item Designed and lead implementation of the cluster-wide performance
harness (telemetry).
\item Data/metadata corruption debugging in distributed system.
\end{itemize}


\section*{VIEWDLE, Kiev -- SENIOR SWE (CV/DSP)}
\label{sec:org1ccafff}
\section*{2009-2011}
\label{sec:orga46a4f7}

Face detection and recognition engine. VIEWDLE was bought by
Google/Motorola Mobility.

Technologies: C++, matlab, opencl, opencv, intel tbb, sse, neon, linux.


\section*{LUXOFT, Odessa --  SENIOR EMBEDDED ENGINEER}
\label{sec:org5c2e852}
\section*{2007-2009}
\label{sec:orgbaa0141}

Emergency Call controller.

Technologies: C++, RTOS, VME, QNX.


\section*{LUXOFT, Odessa -- SWE}
\label{sec:org78913d7}
\section*{2006-2007}
\label{sec:orgf9de538}

Graphical rasterizer library.

Technologies: C++, RTOS, VME, QNX.

\section*{HARDWARE PROJECTS}
\label{sec:orgdd46fc1}
\section*{2002-2006}
\label{sec:org6bdc062}

\begin{itemize}
\item This part of CV does not include precise and full list of completed
projects;
\item Hardware-related experience mostly in airspace and
telecommunication;
\item Software and hardware for sensors and actuators, digital engine
control systems, telemetry systems in airspace domain.
\item Software and hardware components for telephone station switch.

Technologies: AVR, LPC2100, ATSAM4LC4C, MPC555, Altera Cyclone IV
EP4CE6, Xilinx Spartan 3E XC3S500E, PCB manufacturing.
\end{itemize}

\section*{PUBLICATIONS}
\label{sec:orgc6c3e70}
\begin{itemize}
\item US PATENT · \href{https://patents.google.com/patent/US20230035666A1}{US390999147} · ANOMALY DETECTION IN STORAGE SYSTEMS · Issued Feb 2, 2023;
\item US PATENT · \href{https://patents.google.com/patent/US11442715B1/en?inventor=Anatolii+Bilenko}{US11442715B1} · ASYNCHRONOUS FRAMEWORK · Issued Sep 13, 2022;
\item PUBLICATION · \href{https://scholar.google.com/citations?view\_op=view\_citation\&hl=th\&user=j5r-Y28AAAAJ\&citation\_for\_view=j5r-Y28AAAAJ:Y0pCki6q\_DkC}{GRAPH PARTITIONING METHODS FOR COMPUTATIONS IN RECONFIGURABLE SYSTEMS} · Issued 2012;
\item For full list of publications follow to \href{https://scholar.google.com/citations?user=j5r-Y28AAAAJ\&hl=th}{Google Scholar} or PhD thesis \href{https://github.com/just-now/cv/blob/main/aref.pdf}{annotation}.
\end{itemize}

\section*{EDUCATION}
\label{sec:orgc2176b5}
\begin{itemize}
\item Odessa National Polytechnic University, 2001 - 2007, Master of
Science in EECS. GPA: 98/100. Thesis: "Classification of wavelet
functions";
\item Odessa National Polytechnic University, 2008 - 2013,
PhD in EECS. \href{https://github.com/just-now/cv/blob/main/aref.pdf}{Thesis}: "Methods of performance increase in reconfigurable
computing systems by means of new algorithmic and structural
organization".
\end{itemize}
\section*{COMMUNITY}
\label{sec:org6b9de50}
\begin{itemize}
\item \href{https://www.youtube.com/watch?v=FFTi2XNFb7A}{Seagate | Meet the Architect – CORTX Observability with Anatoily Bilenko};
\item \href{https://www.youtube.com/watch?v=ujyIsCt6bbM}{Seagate | Meet the Architect – CORTX DTM: Resiliency in Distributed Systems};
\item Provided lectures on "Processor design" and "\href{https://github.com/just-now/slisp/}{Compiler design}" read
in Odessa National Polytechnic University, 2009-2015, 2021-2023.
My role: volunteer, leader, organizer.
\end{itemize}
\end{document}
